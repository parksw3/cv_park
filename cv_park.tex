%%%%%%%%%%%%%%%%%%%%%%%%%%%%%%%%%%%%%%%%%
% Compact Academic CV
% LaTeX Template
% Version 2.0 (6/7/2019)
%
% This template originates from:
% https://www.LaTeXTemplates.com
%
% Authors:
% Dario Taraborelli (http://nitens.org/taraborelli/home)
% Vel (vel@LaTeXTemplates.com)
%
% License:
% CC BY-NC-SA 3.0 (http://creativecommons.org/licenses/by-nc-sa/3.0/)
%
%%%%%%%%%%%%%%%%%%%%%%%%%%%%%%%%%%%%%%%%%

%----------------------------------------------------------------------------------------
%	PACKAGES AND OTHER DOCUMENT CONFIGURATIONS
%----------------------------------------------------------------------------------------

\documentclass[11pt]{article} % Default document font size

%%%%%%%%%%%%%%%%%%%%%%%%%%%%%%%%%%%%%%%%%
% Compact Academic CV
% Structural Definitions
% Version 1.0 (6/7/2019)
%
% This template originates from:
% https://www.LaTeXTemplates.com
%
% Authors:
% Dario Taraborelli (http://nitens.org/taraborelli/home)
% Vel (vel@LaTeXTemplates.com)
%
% License:
% CC BY-NC-SA 3.0 (http://creativecommons.org/licenses/by-nc-sa/3.0/)
%
%%%%%%%%%%%%%%%%%%%%%%%%%%%%%%%%%%%%%%%%%

%----------------------------------------------------------------------------------------
%	REQUIRED PACKAGES AND MISC CONFIGURATIONS
%----------------------------------------------------------------------------------------

\usepackage{graphicx} % Required for including images

\setlength{\parindent}{0pt} % Stop paragraph indentation

%----------------------------------------------------------------------------------------
%	MARGINS
%----------------------------------------------------------------------------------------

\usepackage{geometry} % Required for adjusting page dimensions and margins

\geometry{
	paper=a4paper, % Paper size, change to letterpaper for US letter size
	top=3.25cm, % Top margin
	bottom=4cm, % Bottom margin
	left=3.5cm, % Left margin
	right=3.5cm, % Right margin
	headheight=0.75cm, % Header height
	footskip=1cm, % Space from the bottom margin to the baseline of the footer
	headsep=0.75cm, % Space from the top margin to the baseline of the header
	%showframe, % Uncomment to show how the type block is set on the page
}

%----------------------------------------------------------------------------------------
%	FONTS
%----------------------------------------------------------------------------------------

\usepackage[utf8]{inputenc} % Required for inputting international characters
\usepackage[T1]{fontenc} % Output font encoding for international characters

\usepackage[semibold]{ebgaramond} % Use the EB Garamond font with a reduced bold weight

%----------------------------------------------------------------------------------------
%	SECTION STYLING
%----------------------------------------------------------------------------------------

\usepackage{sectsty} % Allows changing the font options for sections in a document

\sectionfont{\fontsize{13.5pt}{18pt}\selectfont} % Set font options for sections
\subsectionfont{\mdseries\scshape\normalsize} % Set font options for subsections
\subsubsectionfont{\mdseries\upshape\bfseries\normalsize} % Set font options for subsubsections

%----------------------------------------------------------------------------------------
%	MARGIN YEARS
%----------------------------------------------------------------------------------------

\usepackage{marginnote} % Required to output text in the margin

\newcommand{\years}[1]{\marginnote{\scriptsize #1}} % New command for adding years to the margin
\renewcommand*{\raggedleftmarginnote}{} % Left-align the years in the margin
\setlength{\marginparsep}{-10pt} % Move the margin content closer to the text
\reversemarginpar % Margin text to be output into the left margin instead of the default right margin

%----------------------------------------------------------------------------------------
%	COLOURS
%----------------------------------------------------------------------------------------

\usepackage[usenames, dvipsnames]{xcolor} % Required for specifying colours by name

%----------------------------------------------------------------------------------------
%	LINKS
%----------------------------------------------------------------------------------------

\usepackage[bookmarks, colorlinks, breaklinks]{hyperref} % Required for links

% Set link colours
\hypersetup{
	linkcolor=blue,
	citecolor=blue,
	filecolor=black,
	urlcolor=MidnightBlue
}
 % Include the file specifying the document structure and styling

% Set PDF meta-information
\hypersetup{
	pdftitle={Albert Einstein - Curriculum vitae},
	pdfauthor={Albert Einstein}
}

%----------------------------------------------------------------------------------------

\begin{document}

%----------------------------------------------------------------------------------------
%	CONTACT AND GENERAL INFORMATION
%----------------------------------------------------------------------------------------

{\LARGE\bfseries Sang Woo Park} % Name
\bigskip\bigskip\medskip % Whitespace

Princeton University\\ % Address
Department of Ecology and Evolutionary Biology\\
Princeton, N.J. 08544 USA
\medskip % Whitespace

Email: \href{mailto:swp2@princeton.edu}{swp2@princeton.edu}\\ % Email address

%----------------------------------------------------------------------------------------
%	EDUCATION
%----------------------------------------------------------------------------------------

\section*{Education}

\years{2014-2019}\textsc{BSc} in Mathematics and Statistics (Honours), McMaster University, Hamilton, ON, Canada\\
\years{2019-}\textsc{PhD candidate} in Ecology and Evolutionary Biology, Princeton University, Princeton, NJ, USA\\

%----------------------------------------------------------------------------------------
%	PUBLICATIONS AND TALKS
%----------------------------------------------------------------------------------------

\section*{Publications}

ORCID: 0000-0003-2202-3361. See \href{https://scholar.google.com/citations?user=ZSCrs78AAAAJ&hl=en&oi=ao}{Google Scholar} for links to articles.\\

\years{2022} Sender, R., Bar-On, Y., \textbf{Park, S.W.}, Noor, E., Dushoff, J., and Milo, R., 2022. The unmitigated profile of COVID-19 infectiousness. \textit{eLife}, 11L e79134.

\years{2022} Messacar, K., Baker, R.E., \textbf{Park, S.W.}, Nguyen-Tran, H., Cataldi, J.R., and Grenfell, B., 2022. Preparing for uncertainty: endemic paediatric viral illnesses after COVID-19 pandemic disruption. \textit{Lancet}.

\years{2022} \textbf{Park, S.W.}, Bolker, B.M., Funk, S., Metcalf, C.J.E., Weitz, J.S., Grenfell, B.T., and Dushoff, J., 2022. The importance of the generation interval in investigating dynamics and control of new SARS-CoV-2 variants. \textit{Journal of The Royal Society Interface}, 19: 20220173-20220173\\

\years{2022} Nguyen-Tran, H., \textbf{Park, S.W.}, Messacar, K., Dominguez, S.R., Vogt, M.R., Permar, S., Permaul, P., Hernandez, M., Douek, D.C., McDermott, A.B., Metcalf, C.J.E., Grenfell, B., and Spaulding, A.B., 2022. Enterovirus D68: a test case for the use of immunological surveillance to develop tools to mitigate the pandemic potential of emerging pathogens. \textit{The Lancet Microbe}, 3(2): e83-e85.\\

\years{2022} Lizewski, R.A.\textsuperscript{*}, Sealfon, R.S.G.\textsuperscript{*}, \textbf{Park, S.W.}\textsuperscript{*}, Smith, G.R.\textsuperscript{*}, Porter, C.K.\textsuperscript{*}, Gonzalez-Reiche, A.S.\textsuperscript{*}, Ge, Y.\textsuperscript{*}, Miller, C.M.\textsuperscript{*}, Goforth, C.W., Pincas, H., Termini, M.S., Ramos, I., Nair, V.D., Lizewski, S.E., Alshammary, H., Cer, R.Z., Chen, H.W., George, M.-C., Arnold, C.E., Glang, L.A., Long, K.A., Malagon, F., Marayag, J.J., Nunez, E., Rice, G.K., Santa Ana, E., Schilling, M.A., Smith, D.R., Sugiharto, V.A., Sun, P., van de Guchte, A., Khan, Z., Dutta, J., Vangeti, S., Voegtly, L.J., Weir, D.L., Metcalf, C.J.E., Troyanskaya, O.G., Bishop-Lilly, K.A., Grenfell, B.T., van Bakel, H., Letizia, A.G.\textsuperscript{*}, and Sealfon, S.C.\textsuperscript{*}, 2022. SARS-CoV-2 outbreak dynamics in an isolated US military recruit training center with rigorous prevention measures. \textit{Epidemiology}, 33(6): 797-807.\\
\textsuperscript{*}Contributed equally.\\

\years{2021} Baker, R.E., \textbf{Park, S.W.}, Wagner, C.E., and Metcalf, C.J.E., 2021. The limits of SARS-CoV-2 predictability. \textit{Nature Ecology \& Evolution}, 5(8): 1052-1054.

\years{2021} Dushoff, J., and \textbf{Park, S.W.}, 2021. Speed and strength of an epidemic intervention. \textit{Proceedings of the Royal Society B}, 288(1947): 20201556.\\

\years{2021} \textbf{Park, S.W.}, Pons-Salort, M., Messacar, K., Cook, C., Meyers, L., Farrar, J., Grenfell, B.T., 2021. Epidemiological dynamics of enterovirus D68 in the United States and implications for acute flaccid myelitis. \textit{Science Translational Medicine}, 13(584): eabd2400.\\

\years{2020} \textbf{Park, S.W.}, Sun, K., Champredon, D., Li, M., Bolker, B.M., Earn, D.J.D., Weitz, J.S., Grenfell, B.T. and Dushoff, J., 2020. Forward-looking serial intervals correctly link epidemic growth to reproduction numbers. \textit{Proceedings of the National Academy of Sciences}, 118(2): e2011548118.\\

\years{2020} Weitz, J.S., \textbf{Park, S.W.}, Eksin, C. and Dushoff, J., 2020. Awareness-driven behavior changes can shift the shape of epidemics away from peaks and toward plateaus, shoulders, and oscillations. \textit{Proceedings of the National Academy of Sciences}, 117(51): 32764-32771.\\

\years{2020} Baker, R.E., \textbf{Park, S.W.}, Yang, W., Vecchi, G.A., Metcalf, C.J.E. and Grenfell, B.T., 2020. The impact of COVID-19 nonpharmaceutical interventions on the future dynamics of endemic infections. \textit{Proceedings of the National Academy of Sciences}, 117(48): 30547-30553.\\

\years{2020} Metcalf, C.J.E., Morris, D.H., and \textbf{Park, S.W.}, 2020. Mathematical models to guide pandemic response. \textit{Science}, 369(6502): 368-369.\\

\years{2020} \textbf{Park, S.W.}, Bolker, B.M., Champredon, D., Earn, D.J.D., Li, M., Weitz, J.S., Grenfell, B.T. and Dushoff, J., 2020. Reconciling early-outbreak estimates of the basic reproductive number and its uncertainty: framework and applications to the novel coronavirus (SARS-CoV-2) outbreak. \textit{Journal of the Royal Society Interface}, 17: 20200144.\\

\years{2020} \textbf{Park, S.W.}, Champredon, D., and Dushoff, J., 2020. Inferring generation-interval distributions from contact-tracing data. \textit{Journal of the Royal Society Interface}, 17(167): 20190719.\\

\years{2020} Weitz, J.S., Beckett, S.J., Coenen, A.R., Demory, D., Dominguez-Mirazo, M., Dushoff, J., Leung, C.-Y., Li, G., Măgălie, A., \textbf{Park, S.W.}, Rodriguez-Gonzalez, R., Shivam, S., and Zhao, C.Y., 2020. Modeling shield immunity to reduce COVID-19 epidemic spread. \textit{Nature medicine}, 26(6): 849-854.\\

\years{2020} \textbf{Park, S.W.}, Cornforth, D.M., Dushoff J., and Weitz J.S., 2020. The time scale of asymptomatic transmission affects estimates of epidemic potential in the COVID-19 outbreak. \textit{Epidemics}, 31: 100392.\\

\years{2020} \textbf{Park, S.W.}, and Bolker, B.M., 2020. A note on observation processes in epidemic models. \textit{Bulletin of Mathematical Biology}, 82(3): 1-8.\\

\years{2020} \textbf{Park, S.W.}, Sun, K., Viboud, C., Grenfell, B.T., and Dushoff, J., 2020. Potential Role of Social Distancing in Mitigating Spread of Coronavirus Disease, South Korea. \textit{Emerging infectious diseases}, 26(11): 2697–2700.\\

\years{2019} \textbf{Park, S.W.}, Champredon, D., Weitz, J.S., and Dushoff, J., 2019. A practical generation-interval-based approach to inferring the strength of epidemics from their speed. \textit{Epidemics}, 27: 12-18.\\

\years{2018} \textbf{Park, S.W.}, Dushoff, J., Earn, D.J.D., Poinar, H., and Bolker, B.M., 2018. Human
ectoparasite transmission of the plague during the Second Pandemic is only weakly
supported by proposed mathematical models. \textit{Proceedings of the National Academy of Sciences}, 115(34): E7892-E7893.\\

\years{2017} \textbf{Park, S.W.}, and Bolker, B.M., 2017. Effects of contact structure on the transient
evolution of HIV virulence. \textit{PLoS Computational Biology}, 13(3): e1005453.\\

\years{2017} Rekart, M.L., Ndifon, W., Brunham, R.C., Dushoff, J., \textbf{Park, S.W.}, Rawart, S., and
Cameron, C.E., 2017. A double-edged sword: does highly active antiretroviral therapy contribute to syphilis incidence by impairing immunity to Treponema pallidum?.
\textit{Sexually Transmitted Infections}, 93(5): 374-378.\\

\section*{Preprints}

\years{2022} \textbf{Park, S.W.}, Dushoff, J., Grenfell, B.T., and Weitz, J.S., 2022. Intermediate levels of asymptomatic transmission can lead to the highest levels of epidemic fatalities. \url{https://www.medrxiv.org/content/10.1101/2022.08.01.22278288v1}\\

\years{2022} \textbf{Park, S.W.}, Sun, K., Abbott, S., Sender, R., Bar-On, Y.M., Weitz, J.S., Funk, S., Grenfell, B.T., Backer, J.A., Wallinga, J., Viboud, C., and Dushoff, J., 2022. Inferring the differences in incubation-period and generation-interval distributions of the Delta and Omicron variants of SARS-CoV-2. \url{https://www.medrxiv.org/content/10.1101/2022.07.02.22277186v1}.\\

\years{2022} Harris, J.D., \textbf{Park, S.W.}, Dushoff, J., and Weitz, J.S., 2022. How time-scale differences in asymptomatic and symptomatic transmission shape SARS-CoV-2 outbreak dynamics. \url{https://www.medrxiv.org/content/10.1101/2022.04.21.22274139v1}\\

\years{2022} Lee, W.E., \textbf{Park, S.W.}, Weinberger, D.M., Olson, D., Simonsen, L., Grenfell, B.T., and Viboud, C., 2022. Direct and indirect mortality impacts of the COVID-19 pandemic in the US, March 2020-April 2021. \url{https://www.medrxiv.org/content/10.1101/2022.02.10.22270721v1}\\

\section*{Software}

\years{2022} \textbf{Park, S.W.}, and Bolker, B.M., \texttt{fitode}: Tools for Ordinary Differential Equations Model Fitting. \url{https://cran.r-project.org/web/packages/fitode/index.html}.

\section*{Teaching}

\years{2019--2021} \textbf{Graduate Assistantship in Instruction}, Princeton University.
\begin{itemize}
  \item Disease Ecology, Economics, and Policy (ENV 304 / ECO 328 / EEB 304 / SPI 455), Fall 2019, 2020, 2021.
\end{itemize}

\section*{Textbook}

\years{2017} Alama S., and \textbf{Park. S.W.}, MATH 2XX3 -- Advanced Calculus II: Class notes recorded, adapted, and illustrated by Sang Woo Park. Available at McMaster University Library.

\section*{Professional service}

Manuscript reviewer for PNAS, PLOS Computational Biology, BMC Medicine, Emerging Infectious Diseases, Proceedings of the Royal Society A, Proceedings of the Royal Society B, Epidemics, Scientific Reports, Mathematical Biosciences, PLOS One, PeerJ, etc.

%----------------------------------------------------------------------------------------
%	GRANTS, HONOURS AND AWARDS
%----------------------------------------------------------------------------------------

%----------------------------------------------------------------------------------------
%	TEACHING
%----------------------------------------------------------------------------------------

\vfill % Whitespace before final footer

%----------------------------------------------------------------------------------------
%	FINAL FOOTER
%----------------------------------------------------------------------------------------

% Any final footer text such as a URL to the latest version of this CV, last updated date, compiled in XeTeX, etc
\begin{center}
	\scriptsize
	Last updated: \today
\end{center}

%----------------------------------------------------------------------------------------

\end{document}
